\documentclass[a4paper,12pt]{article}

\usepackage{url}
\usepackage[utf8x]{inputenc}
\usepackage[ngerman]{babel}
\usepackage[colorlinks=true, pdfborder={0 0 0}, pdftex, breaklinks=true, linkcolor=blue, citecolor=orange , urlcolor=blue, linktocpage=true]{hyperref}
\usepackage{geometry}
\geometry{left=25mm, right=30mm, top=30mm, bottom=35mm}

\begin{document}

\title{
\textbf{Technische Abschätzung}\\ 
CCS Projekt - Multicoptersteuerung
}
\author{Michael Wörner \and Christian Silfang}
\date{}

\parskip1.5ex
\parindent0em

\maketitle

Vor Bestellung der Hardware und den dazugehörigen Komponenten wurden Abschätzungen mit Hilfe von Anfragen an den Support des Flyduino-Shops gemacht.

\noindent\rule[1ex]{\textwidth}{1pt}
\vspace{1cm}

\textbf{Wie groß kann die Traglast für den Nanoquad sein?}

Antwort Flyduino-Support:

Bis 50g ist ein sportlicher Flug möglich. Ab 100g bleibt der Nanoquad flugfähig kommt jedoch in den kritischen Bereich ab 100g aufwärts.

\textbf{Wie lange hält ein 2S LiPo-Akku mit 850 mAh?} 

Antwort Flyduino-Support:

Für 15 Minuten schwebend mit leichten Bewegungen. 5 Minuten unter Volllast, wie bspw. während dem Kunstflug.

\end{document}