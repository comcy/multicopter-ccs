\documentclass[12pt,a4paper]{article}
\usepackage[utf8]{inputenc}
\usepackage[german]{babel}
\usepackage{amsmath}
\usepackage{amsfonts}
\usepackage{amssymb}
\usepackage{graphicx}
\usepackage[left=2cm,right=2cm,top=2cm,bottom=2cm]{geometry}
\title{Multicoptersteuerung Projektziele}
\begin{document}
\maketitle

\begin{center}
	\author{Michael Wörner 30137, \and Christian Silfang 30147}
\end{center}

\section{Ziel und Beschreibung der Arbeit}
	Flugfähiger Multicopter der über verschiedene Schnittstellen gesteuert werden soll.
	Die hierfür benötigten Schnittstellen und dazugehörigen Protokolle soll ebenfalls im Rahmen dieser Arbeit 
	evaluiert werden.
	
	Dabei werden vorgefertigte Teilkomponenten für den Mechatronischen Aufbau verwendet. Diese werden 
	im wesentlichen folgende sein:
	\begin{itemize}
		\item	Rahmen
		\item	Ausleger
		\item	Flugcontroller
		\item	Motoren und Rotoren 
		\item	Motorsteuerung (ESC)
		\item	Akku (Lipo)
		\item	Sender und Empfänger\footnote{DSM 2,4GHz, Zigbee, Bluetooth, etc.}
	\end{itemize}
	
	In erster Linie sollen verschiedene Sensoren, Schnittstellen und Protokolle am selbst entwickeltem flugfähigem Objekt erprobt und 
	optimiert werden. Ziel ist es die Flugsteuerung so zu erweitern, dass das System selbstständig agieren kann.

\section{Meilensteine}
	\begin{itemize}
		\item	Planung
		\item	Beschafung und Aufbau der Hardware
		\item	Reglerentwurf Flugstabilisierung
		\item	Start der Hauptaufgabe (Abschluss-Ziel)
		\item 	\textbf{Zwischenpräsentation} 
		\item 	Weiterentwicklung und Optimierung der definierten Ziele
		\item 	Projektabschluss
		\item 	\textbf{Abschlusspräsentation} 
	\end{itemize}
	
	Mögliche Umsetzungen zur Definition eines Abschluss-Ziels können im wesentlichen erst während der Arbeit abgeschätzt werden. 
	Aufgrund der technischen Gegebenheiten, die den Aufbau des Multicopters beeinflussen, muss der Rahmen des Verwendungszweck vor Projektbeginn 
	festgelegt werden. Dieser soll auf den Indoor-Bereich festgelegt werden um zunächst äußerliche Störfaktoren und Probleme mit dem Gesetzgeber zu 
	vermeiden. Eine autonome arbeitsweise steht hierbei im Vordergrund. Konkrete Ziele können nach ersten Tests klarer definiert werden. Bevor mit der 
	eigentlichen Aufgabe begonnen wird, sollen Teilziele erreicht werden:
	\begin{itemize}
		\item korrekte Arbeitsweise der Mechatronik
		\item manuelles Fliegen		
		\item Flugstabilisation auf jeder Achse
		\item Position-Holdings
		\item teil-autonome Ausführungen (bspw. Befehl, mit einem Schub nach rechts zu driften/Anstieg bis auf gewisse Höhe) 
	\end{itemize}
	
	Nach Erreichen der Teilziele, wären folgende Umsetzungen denkbar:	
	\begin{itemize}
		\item 	Flug von Punkt zu Punkt
		\item 	Flug einer einfachen geometrischer Form im Raum
		%\item	ausschöpfung von Sensoren (Kollisionserkenung) 
		\item	evtl. automatisches aufnehmen, Transport und ablegen von Objekten (bswp. ein Dominostein)
	\end{itemize} 


\end{document}
