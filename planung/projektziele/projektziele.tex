\documentclass[12pt,a4paper]{article}
\usepackage[utf8]{inputenc}
\usepackage[german]{babel}
\usepackage{amsmath}
\usepackage{amsfonts}
\usepackage{amssymb}
\usepackage{graphicx}
\usepackage[left=2cm,right=2cm,top=2cm,bottom=2cm]{geometry}
\title{Multicoptersteuerung Projektziele}
\begin{document}
\maketitle

\section{Ziel und Beschreibung der Arbeit}
	Flugfähiger Multicopter der über verschiedene Schnittstellen gesteuert werden soll.
	Die hierfür benötigten Schnittstellen und dazugehörigen Protokolle soll ebenfalls im Rahmen dieser Arbeit 
	evaluiert werden.
	
	Dabei werden vorgefertigte Teilkomponenten für den Mechatronischen Aufbau verwendet. Diese werden 
	im wesentlichen folgende sein:
	\begin{itemize}
		\item	Rahmen
		\item	Ausleger
		\item	Flugcontroller
		\item	Motoren und Rotoren 
		\item	Motorsteuerung (ESC)
		\item	Akku (Lipo)
		\item	Sender und Empfänger\footnote{DSM 2,4GHz, Zigbee, Bluetooth, etc.}
	\end{itemize}
	
	In erster Linie sollen verschiedene Sensoren, Schnittstellen und Protokolle am selbst entwickeltem flugfähigem Objekt erprobt und 
	optimiert werden. Ziel ist es die Flugsteuerung so zu erweitern, dass das System selbstständig agieren kann.

\section{Meilensteine}
	\begin{itemize}
		\item	Planung
		\item	Beschafung und Aufbau der Hardware
		\item	Reglerentwurf Flugstabilisierung
		\item	finden einer Sinvollen Aufgabe
	\end{itemize}
	
\section{Beschreibung}
	\subsection{Vorgehensweise}
%		\begin{itemize}
%			\item	Ermittelung von Meilensteinen mit grober Abschätzung des Zeitaufwands
%			\item	Entwurf des Multicopter-Aufbaus\footnote{siehe Ziel der Arbeit}
%			\item	Evaluierung diverser Schnittstellen und Sensoren
%		\end{itemize}
%	
	
%	\subsection{Aufbau}
%		\begin{itemize}
%			\item	Vorgesehen werden vier Antriebe
%			\item	
%		\end{itemize}



\end{document}