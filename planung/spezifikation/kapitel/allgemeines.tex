\section{Allgemeines}
\subsection{Kurze Beschreibung}
	Es soll ein Flug fähiger Multicopter mit vorzugsweise 4 Rotoren entwickelt werden der in der Lage ist,
	in geschlossenen Räumen autonom zu agieren. 

	Die endgültigen Aufgaben sollen im Laufe der Entwicklung ermittelt und festgelegt werden werden.
	
\subsection{Entwicklungsablauf}
	\subsubsection{Beteiligte Personen}
		
	
	
\subsection{Dokumentation}
	\subsubsection{Allgemeine Bestimmung}
		Als Nachweis für die Einhaltung der Spezifikationsvorgaben ist eine umfassende 
		Dokumentation zu erstellen. Werden keine anderen Vereinbarungen getroffen,
		sind alle Dokumente im Portable Document Format (PDF) anzufertigen.
	
	\subsubsection{Systemmodell}
		Es ist \textbf{wünschenswert}, dass für das Entwickelte System ein Modell erstellt wird.
		Dies soll mittels der Systemmodelierungssprache \textit{SysML} realisiert werden.


\subsection{Verwendete Werkzeuge}
	\subsubsection{Versionsverwaltung}
		Für Versionskontrolle und den gemeinsamen Datenaustausch wird \textit{git}, mit einem 
		gemeinsamen Repository auf \url{https://www.github.com}, verwendet werden.
		Personen die aus diesem Repository lesen und schreiben wollen müssen dies mit Herrn 
		Christian Silfang persönlich absprechen.
		
	\subsubsection{Werkzeuge für die Dokumentation}
		Alle selbst verfassten Unterlagen die im Laufe diese Projektes zur Dokumentation und Plannung 
		entstehen sollen in Latex angefertigt werden. 
		
		Kreative Arbeitsprozesse sollen mittels Mindmaps verrichtet werden. Wie diese erstellt werden soll den
		jeweiligen Personen selbst überlassen werden. Jedoch sollen die Ergebnisse mithilfe des Programms 
		\textit{Freemind} in eine Form gebracht werden die für alle zugänglich ist.
		
	\subsubsection{Systemmodellierung}
		Für die Systemmoldellierung in SysML soll das Programm \textit{Topcased}\footnote{Eclipse Plugin}
		verwendet werden.
		
	\subsection{Mathematische Berechnungen}
		Damit komplexe bzw. aufwendige Berechnung effektiv und schnell druchgeführt und für alle gleich zugänglich
		gemacht werden, soll das Programm \textit{wxMaxima} verwendet werden.
		
		Für nummerische Berechnungen bzw. Optimierungen soll \textit{Scilab} und die dazugehörigen Unterprogramme
		verwendet werden.
		
		Zur Graphischen Darstellung von zwei- und dreidimensionalen Funktionen soll \textit{Gnuplot} verwendet werden.
		Dies ist auch ein Bestandteil von wxMaxima und Scilab.
	
	