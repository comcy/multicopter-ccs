\documentclass[a4paper,12pt]{article}

\usepackage{url}
\usepackage[utf8x]{inputenc}
\usepackage[ngerman]{babel}
\usepackage[colorlinks=true, pdfborder={0 0 0}, pdftex, breaklinks=true, linkcolor=blue, citecolor=orange , urlcolor=blue, linktocpage=true]{hyperref}
\usepackage{geometry}
\geometry{left=25mm, right=35mm, top=30mm, bottom=35mm}

\begin{document}


\title{
\textbf{Protokoll}\\
CCS Projekt: Multicoptersteuerung}
\author{Michael Wörner \and Christian Silfang \and Frank Kleesattel}
\date{}

\parskip1.5ex
\parindent0em

\maketitle

Protokoll des Projekts \glqq Multicoptersteuerung\grqq. Der organisatorische sowie technische Verlauf des Projekts soll in diesem Dokument festgehalten werden.

\noindent\rule[1ex]{\textwidth}{1pt}
\vspace{1cm}

\textbf{07.04.2014}\\
Kickoff-Treffen mit Prof. Dr. Bantel. Reultat des Treffens war die Einigung auf die Entwicklung einen Nanoquad. Hierzu soll im Folgenden eine Bestellliste aller erforderlichen Teile angefertigt werden.

\textbf{08.04.2014}\\
Technische Abschätzung angefangen. Hierzu wurden Fragen an den Support des Flyduino-Shops formuliert.
Bestellliste mit Auswahl der Kompnenten für das Projekt begonnen.

\textbf{10.04.2014}\\
Aufnahme eines dritten Teammitgliedes - Frank Kleesattel. Dies wurde mit Herrn Prof. Bantel
persönlich besprochen. Die Anmeldung wurde am selben Tag im Sekretariat abgegeben. Diese musste 
auf den 07.04.2014 zurückdatiert werden, da dies der letztmögliche Anmeldetermin war. Dies geschah
unter der Absprache von Herrn Bantel und Frau Lange.

\textbf{14.04.2014}\\
Bestellung an Herrn Bantel übergeben.

\textbf{27.05.2014}\\
Bestellung von Flyduino erhalten.

\textbf{12.06.2014}\\
Beginn des Zusammenbaus der einzelnen Komponenten des Multicopters.

\textbf{12.06.2014}\\
Beginn des Zusammenbaus der einzelnen Komponenten des Multicopters.
\newpage

\textbf{16.06.2014 - 08.08.2014}\\
Prüfungs- und Lernphase. Anschließend Urlaub in Vorlesungsfreier Zeit.


\end{document} 
