\documentclass[12pt,a4paper]{scrartcl}

%Eingebundene Allgemeine Pakete
\usepackage[utf8]{inputenc}
\usepackage[german]{babel}
\usepackage{amsmath}
\usepackage{amsfonts}
\usepackage{amssymb}
\usepackage{graphicx}
\usepackage{framed}
\usepackage{pdfpages}
\usepackage{subfigure}
\usepackage[left=3cm,right=2cm,top=2.5cm,bottom=1.5cm]{geometry}

%\usepackage[colorlinks=true, linkcolor=black]{hyperref}
\usepackage[colorlinks=true, allcolors=blue]{hyperref}

  
\usepackage{caption}
\usepackage{listings}
%\usepackage[hidelinks]{hyperref}

\usepackage{wrapfig}
%Spezeille Pakete und einstellungen

%=============================
% BibLatex
%=============================
%\usepackage[style=numeric]{biblatex}
\usepackage[style=numeric]{biblatex} 
\addbibresource{bibtex/literatur}

%============================================
%----------------Seitenlayout----------------
%============================================
\usepackage{fancyhdr}
\pagestyle{fancy}
\fancyhead{} %erst mal leer machen
\fancyfoot{} % - '' -

\fancyhead[L]{\rightmark}
\fancyhead[R]{\thepage}
\renewcommand{\headrulewidth}{1pt}

\begin{document}

\parindent0mm	%kein einzug bei Absatzbegin
%===========================================================%
%					    Titelseite
%===========================================================%
\pagenumbering{Roman}
\begin{titlepage}
	\rule{1\textwidth}{2pt}
	\vspace*{1.5cm}
	\begin{center}
	\LARGE{
		\textbf{}
		}

	\vspace*{1.5cm}
	
	\LARGE{\textbf{Multicoptersteuerung}}
	\vspace*{1.5cm}
	
	Projektarbeit
	
	\vspace*{1.5cm}
	
	von
	\vspace*{1.5cm}
	
	Frank Kleesattel (30000)\\
	Christian Sielfang (30147)\\
	Michael Wörner (30137)

	\vspace*{1.5cm}

	Betreuer:\\
	Prof. Dr. Winfried Bantel
	
	\vspace*{2.5cm}
	
	Hochschule Aalen
	
	Hochschule für Technik und Wirtschaft
	
	UNIVERSITY OF APPLIED SCIENCES
	
	\vspace*{1.0cm}
	\normalsize
	
%	\begin{tabular}{ll}
%		%Betreuender Dozent: & Prof. Dr. Wilhelm Kleppmann\\
%		%Studiengang: & Computer Controlled Systems
%	\end{tabular}
	\end{center}
	\vfill
	\rule{1\textwidth}{2pt}
	
\end{titlepage}
\pagenumbering{arabic}
%===========================================================%
%					    Text
%===========================================================%
\tableofcontents
\newpage
\input{kapitel/einleitung}
\section{Grundlagen}
Dieser Abschnitt der Ausarbeitung befasst sich mit den Grundlagen die für das Verständnis des gesamten Projekts nötig sind. Zu Beginn sollen dem Leser die Grundlagen des Fliegens allgemein und mit Multicoptern näher gezeigt werden.

Im darauf folgenden Unterabschnitt soll grundlegendes Wissen und die verschiedenen Konzepte des Multicopterbaus näher erklärt werden. Zunächst soll der Basisaufbau eines jeden Multicopters mit seinen einzelnen Komponenten dargestellt werden. Vor allem die Konzeption zum Bau eines Multicopters, mit allen verwendeten Teilkomponenten die für das Projekt nötig waren, nehmen einen großen Stellenwert in diesem Grundlagenkapitel ein.

Im letzten Unterabschnitt sollen dem Leser kurze, aber dennoch wichtige, rechtliche Hinweise für den Umgang mit Flugobjekten dieser Art gegeben werden. Die Wichtigkeit dieses Themas darf hierbei nicht unterschätzt werden. So waren ebenfalls die Planung und die Umsetzung des hier vorliegenden Projekts maßgeblich davon beeinflusst.

\subsection{Einführung in das Fliegen mit Multicoptern}



\subsection{Konzepte des Multicopterbaus}



\subsubsection{Die Stromquelle}



\subsubsection{Technischer Aufbau eines Multicopters}



\subsubsection{Flugcontroller}



\subsubsection{ESCs und Motoren}



\subsubsection{Sender und Empfänger}



\subsection{Rechtliche Hinweise im Umgang mit Multicoptern}

\input{kapitel/zusammenbau}


%===========================================================%
%					    Quellenverzeichnis
%===========================================================%
\printbibliography[title=Quellenverzeichnis, heading=bibintoc]
\end{document}